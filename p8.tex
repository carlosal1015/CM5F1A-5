\section{Pregunta N$^{\circ}$8\qquad Carlos Alonso Aznarán Laos}

\begin{frame}[fragile]
    \begin{enumerate}\setcounter{enumi}{7}
        \item


              Sea $\Omega=\left(0,L\right)\subset\mathbb{R}$. Obtenga
              la formulación variacional o débil del problema
              de valor de frontera de orden cuatro

              \begin{equation*}
                  \begin{cases}
                      {\left(
                          \alpha\left(x\right)
                              {u\left(x\right)}^{\prime\prime}
                      \right)}^{\prime\prime}=f\left(x\right) &
                      \text { para }
                      x\in\left(0,L\right).                                                       \\
                      u\left(0\right)=
                      u\left(L\right)=0
                                                              & \text { para }x\in\partial\Omega. \\
                      u^{\prime}\left(0\right)=
                      u^{\prime}\left(L\right)=
                      0                                       &
                      \text { para }x\in\partial\Omega.
                  \end{cases}
              \end{equation*}

    \end{enumerate}

    \begin{solution}
        % Para encontrar la formulación variacional del PVF, definir el problema y los espacios funcionales involucrados.

        Sean $\left(L^{2}(\Omega),\left\|\cdot\right\|_{L^{2}(\Omega)}\right)$ un espacio de Hilbert real y
        \begin{math}
            V\coloneqq
            \left\{
            u\mid
            u,
            u^{\prime},
            u^{\prime\prime}\in
            L^{2}(\Omega)
            \right\}
        \end{math}.
        La formulación débil se obtiene multiplicando la ecuación por
        las funciones test $v\in V$ e integrando sobre $\Omega$.
        Luego integramos por partes para manejar las derivadas.


        La forma débil viene dada encontrando $u\in V$ tal que $\forall \alert{v} \in V:$

        \begin{equation*}
            \int_{0}^{L}
            {
            \left(
            \alpha\left(x\right)
                {u\left(x\right)}^{\prime\prime}
            \right)
            }^{\prime\prime}
            \alert{v}\dl x =
            \int_{0}^{L}
            f\left(x\right)\cdot
            \alert{v}
            \dl x.
        \end{equation*}

        Ahora, integra por partes varias veces:

        \begin{align*}
            \int_0^L (\alpha u''')'' v \,dx & = -\int_0^L (\alpha u''')'v' \,dx + [\alpha u''')'v]_0^L                                 \\
                                            & = \int_0^L (\alpha u''')'v' \,dx + \int_0^L \alpha' u'''v' \,dx + [\alpha u'')'v']_0^L   \\
                                            & = -\int_0^L (\alpha' u''')'v \,dx + \int_0^L \alpha'' u'''v \,dx + [\alpha' u''')v']_0^L \\
                                            & = \int_0^L (\alpha' u''')'v \,dx + \int_0^L \alpha'' u'''v \,dx + [\alpha' u'')v]_0^L
        \end{align*}
    \end{solution}
\end{frame}

\begin{frame}
    \begin{solution}
        \begin{align*}
             & = -\int_0^L (\alpha'' u''')v \,dx + \int_0^L \alpha''' u''v \,dx + [\alpha'' u'')v']_0^L \\
             & = \int_0^L \alpha'' u'''v \,dx + \int_0^L \alpha''' u''v \,dx + [\alpha'' u')v]_0^L      \\
             & = -\int_0^L \alpha''' u''v \,dx + \int_0^L \alpha'''' u v \,dx + [\alpha''' u)v']_0^L    \\
             & = \int_0^L \alpha''' u''v \,dx + \int_0^L \alpha'''' u v \,dx - [\alpha''' u)v']_0^L.
        \end{align*}

        Ahora, sustituye esto nuevamente en la forma débil:

        \begin{equation*}
            \int_0^L \alpha''' u''v \,dx + \int_0^L \alpha'''' u v \,dx - [\alpha''' u)v']_0^L = \int_0^L f \cdot v \,dx.
        \end{equation*}

        Finalmente, aplique las condiciones de frontera \( u(0) = u(L) = u'(0) = u'(L) = 0 \) para eliminar los términos de frontera, lo que lleva a la formulación variacional débil:

        \begin{equation*}
            \int_0^L \alpha''' u''v \,dx + \int_0^L \alpha'''' u v \,dx = \int_0^L f \cdot v \,dx.
        \end{equation*}

        Esta es la formulación variacional del PVF de cuarto orden dado.
    \end{solution}
\end{frame}